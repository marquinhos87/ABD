\chapter{Descrição/perceção do Problema} \label{chap:contex}

\hspace{5mm} Neste capítulo faz-se uma descrição/contextualização do problema em análise, abordando-se os objetivos pretendidos.

\section{Objetivos}

\hspace{5mm} Os principais objetivos deste projeto concentram-se na configuração do benchmark TPC-C e de uma máquina para correr o mesmo, sendo esta a configuração de referência. Depois de obtida a configuração de referência serão alteradas as configurações da Base de Dados PostgreSQL (base de dados utilizada para guardar toda a informação do benchmark TPC-C) de forma a tentar obter melhorias no desempenho da configuração de referência.

%Mudar 'Instalação e Configuração' para 'Instalação' porque a configuração faz parte do desenvolvimento?
\section{Instalação e Configuração}

\hspace{5mm} Para a instalação do EscadaTPC-C foi utilizada a documentação fornecida pelo docente da Unidade Curricular, tendo sido necessário a instalação de outras peças de software durante a instalação deste. Para um total funcionamento da aplicação é necessário instalar uma base de dados (MySQL, Derby ou PostgreSQL), sendo que para este projeto foi utilizada a base de dados PostgreSQL.